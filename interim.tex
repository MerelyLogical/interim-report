\documentclass[journal]{IEEEtran}

\usepackage{cite}
\usepackage{float}
\usepackage{graphicx}
\usepackage[caption=false]{subfig}
  \DeclareGraphicsExtensions{.png}
\usepackage{amsmath}
\usepackage{amsfonts}
\usepackage{url}

\hyphenation{photo-graphed diff-erence}

\begin{document}

\title{%
  High-radix On-line Arithmetic Verification System\\
  \large Final Year Project 1800478: Interim Report}
\author{Zifan Wang, 01077639\\Imperial College London}

% \markboth{2018-2019}{?}

\maketitle

% \begin{abstract}
% \end{abstract}

\section{Introduction}
% The Interim report should normally be between 10 and 30 pages long. Remember
% that it must be a project planning document for the remainder of your project
% work, as well as being an early write-up of your background material. Using
% Interim Report material, without explicit reference, in your Final Report is
% allowed, and even encouraged. The Interim Report is often a first draft of the
% background section of the Final Report. Self-plagiarism, in this case, is not an
% issue. It is expected that you will reuse Interim Report material without
% explicit reference.

This project is a part of a larger project investigating the effect of using
high-radix number systems with on-line arithmetic operators.
The overarching aim would involve implementing such a system on FPGA and
quantifying its performance improvements.
This aim would be achieved through two vertically split individual projects.
One would design and verify the arithmetic operator modules,
while the other would design a system from the top-level to test and
evaluate these operators.
This project deals with the system-level issues.


\section{Project Specification}
% The project specification should state clearly what the project is intended to
% deliver, including all hardware, software, simulation, and analytical work, and
% provide some motivation.

At the end of the project, the system should be able to perform the following:
1.Takes in the arithmetic modules designed by the sister project as its input;
2.Generate and run tests on these modules;
3.Vary the frequency and voltage of the FPGA;
4.Evaluate its performance.

\subsection{Project Interfacing}
Numbering system
I/O

\subsection{Hardware Choice}
The system itself will be built on a Cyclone V board. Its SoC has integrated an HPS (Hard processor system) and an FPGA.

\subsection{Deliverables}
The initial deliverable for the engineering side involves running a simple program on the FPGA through the HPS with the FPGA frequency being controllable. After this, the next critical step would be making sure the modules under test will be the point of failure and not the testbench. This would include some research on ways in improving the speed of feeding inputs to the arithmetic units, and checking its outputs. Once this could be confirmed, we can start adding a selection of different functionalities.

1.Running standard benchmarks;
2.Running key algorithms or their components;
3.Experiment with other power efficiency improving techniques, such as undervolting;
4.Add support for different radix arithmetic;
5.Allow graceful failures for the testbench in case of unintended behaviour for the arithmetic modules;
6.Add an interactive UI to control the voltage and frequency at run time and examine the DUT’s behaviour;

Depending on the time situation, more or less items on this list may be fulfilled.

\section{Background}
% The background section must outline the necessary background to the project,
% stating how it is important for the project work. For example: survey of related
% literature, analysis of competing products, technical specifications of hardware
% or software standards, electronic components, necessary software tools,
% background theory. The contents of this section will vary for different
% projects, and in many cases the background reading will have been completed –
% but if not you should be in a position to list what remains to be done (e.g. a
% set of research papers to read and understand). A good benchmark of progress
% here is that you have accumulated (though may not yet have read) at least 20
% references to background material. In projects which have substantial background
% writing up your literature survey in the Interim Report will save time at the
% end of the project and allow this element of your final report to receive timely
% feedback from your supervisor so the final write-up can be improved.

% explain online
% explain high-radix

\section{Implementation Plan}
% The implementation plan is a preliminary breakdown of the work that is to be
% done in the remainder of the project. You should identify a set of milestones
% and provide a realistic estimate of when each of these should be completed if
% all goes well. It should also detail fallback positions in case any stage of the
% development goes wrong. You may feel, in the early stages of your project work,
% that the times in this plan are guesses. However you will find as the project
% progresses that keeping track of and revising your initial estimates, and if
% necessary altering the proposed work, is a vital way to ensure that the project
% is finished in time. In projects with heavy implementation content you should
% document what you have already completed.

Compared to existing research, the combination of high-radix and online is relatively novel, and to optimising it on a system level for popular FPGA accelerations such as neural networks would be innovative as well. With respect to the testbench, figuring out a system to obtain the optimal point of operation of the FPGA, in terms of its voltage and frequency, would also require some advanced research.

\subsection{Risks}
A major risk of this project is related to its schedule and the existence of an initial blocking task. While most of the later sections of the project can be selectively added or removed from the scope relatively easily, the initial setup of the testbench will always remain critical to any further improvements. It is thus vital that the bare minimum system gets done early. To ensure this happens, it will be placed in the highest priority before its completion, and any blocking issue should be discussed with the supervisor if it could not be resolved after significant effort.

The other major risk has to do with the progress of the sister project. The purpose of the testbench is to verify and stress the arithmetic designs. If these designs would not be available near the end of this project, it would be difficult to empirically prove the capabilities of the testbench and its surrounding system. It is not impossible, as there are still substitutions for them. For functional purposes, standard off-the-shelf adders and multipliers could be used in lieu. For other purposes, it is possible to have a model done before the actual design starts in the paired project. While this would allow this project to progress easier, it would be extra work for the other project, which is ultimately up to the decision of the other student. In all, it would be nice to have a solid arithmetic module completely to run in this testbench, but without one, the system can still be built and completed, albeit generating less useful data towards the overall aim of the project.


\section{Evaluation Plan}
% The evaluation plan should detail how you expect to measure the success of the
% project. In particular it should document any tests that are required to ensure
% that the project deliverable(s) function correctly, together with (where
% appropriate) details of experiments required to evaluate the work with respect
% to other products or research results.

\section{Ethical, Legal, and Safety Plan}
% The Ethical, Legal and Safety Plan must detail what are the issues in this are
% relevant to your project, showing how you will comply with best practice. If
% there are no such issues (the case for 80\% of all projects) you must
% nevertheless show here that you have considered these issues and detail why they
% will not apply to your project. Information will be provided on the project web
% pages about Ethical, Legal and Safety matters.

\section{Conclusion}

\newpage
\appendices

\section{Formulae}
$$1 + 1 = 2$$

\section{Data}

\begin{table}[H]
  \centering
  \begin{tabular}{c|cccc}
    Sum             & $1$ & $2$ & $3$ & $4$ \\
    \hline
    $1$             & $2$ & $3$ & $4$ & $5$ \\
    $2$             & $3$ & $4$ & $5$ & $6$ \\
  \end{tabular}
\end{table}

\section{Illustrations}

\begin{figure}[H]
  \centering
  \includegraphics[width=8cm]{img/placeholder}
  \caption{Placeholder}
  \label{ph}
\end{figure}

\begin{thebibliography}{1}

\bibitem{paper1}
    A. Bellet et al,
    ``\textit{A Survey on Metric Learning for Feature Vectors and Structured Data}''
    2013.
    Available at: \url{arxiv.org/abs/1306.6709}.

\bibitem{sklearn}
    F. Pedregosa et al,
    ``\textit{Scikit-learn: Machine Learning in Python}''
    2011.
    Available at: \url{scikit-learn.org}.
    
\bibitem{ML}
    C. J. Carey, Y. Tang,
    ``\textit{metric-learn: Metric Learning in Python}''
    2015.
    Available at: \url{metric-learn.github.io/metric-learn/}.

\bibitem{pyDML}
    J. L. S. Diaz,
    ``\textit{pyDML}''
    2018.
    Available at: \url{pydml.readthedocs.io/en/latest/}.
\bibitem{map}
    J. Hui,
    ``\textit{mAP (mean Average Precision) for Object Detection}''
    2018.
    Available at: \url{medium.com/@jonathan_hui/map-mean-average-precision-for-object-detection-45c121a31173}.

\bibitem{LA}
    R. E. Burkard, E. Cela,
    ``\textit{Linear Assignment Problems and Extensions}''
    Available at: \url{pdfs.semanticscholar.org/f202/410bbc322784673f707dd35cd3220d4bca47.pdf}.

\bibitem{HA}
    H. W. Kuhn,
    ``\textit{The Hungarian Method for the Assignment Problem}''
    1995.
    Available at: \url{onlinelibrary.wiley.com/doi/epdf/10.1002/nav.3800020109}.

\bibitem{lmnn}
    Kilian Q. Weinberger
    ``\textit{Distance Metric Learning for Large Margin
    Nearest Neighbor Classification}''
    2009
    Available at: \url{jmlr.csail.mit.edu/papers/volume10/weinberger09a/weinberger09a.pdff}.

\bibitem{MMC}
    E. P. Xing et al,
    ``\textit{Distance Metric Learning, with Application to Clustering with Side-Information}''
    Available at: \url{www.cs.cmu.edu/~epxing/papers/Old_papers/xing_nips02_metric.pdf}.

\end{thebibliography}

\end{document}